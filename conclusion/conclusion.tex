\section{Conclusion} \label{sec:conclusion}
\subsection{Summary} \label{sec:summary}




% In this report, we have illustrate how SpiNNaker hardware can be applied to Computational fluid dynamics problem. More specifically, a basic implementation as well as some communication optimization of lattice Boltzmann method has been discussed. After confirm the correctness and quantify the accuracy of our implementation, we observe a good weak-scaling with the SpiNNaker hardware. Though there is still some limitation, this project build a solid foundation for future's exploration of porting CFD models into SpiNNaker platform.
% What was easy? What was difficult (compared to e.g., the CPU
% version)? Is SpiNNaker really a good platform for running
% such a simulation? Is it easy to develop and debug? Is the
% communication easy to arrange? Is it easy to measure and
% understand performance? Lattice Boltzmann is relatively
% simple (it was selected for this project to be so). What
% might happen if a more complex model was required?3
In this report, we have illustrate how SpiNNaker hardware can be applied to Computational fluid dynamics problem. More specifically, a basic implementation as well as some communication optimization of lattice Boltzmann method has been discussed. \\

Across the project, the most straightforward part is porting the computation steps from the C implementation to SpiNNaker. This step is largely similar to porting an serial application to MPI, which parallel programmers are usually more familiar with. The hardest part is getting used to the concepts and programming model of SpiNNaker. Unlike the other library aiming at parallel programming, e.g. MPI or OpenMP, SpiNNaker, originally as a neural simulation platform, has a few unique features for parallel computing such as multicast communication and combining python scripts and C language. Though it might be non-trivial for starters, as as the study progresses, developers would find the SpiNNaker project is well-engineered and the communication is easy to handle with the provided APIs.\\

Though the development is becoming straightforward as the developers are getting more familiar with the APIs, debugging is always painful as the other parallel programming techniques. What make debugging even more time-consuming is that when the simulation scale is large the developers have to wait for the long-time data loading for a single round of debugging. Besides, the developers need to take care at both the front end scripts in Python and source code in C, which make the debugging even more complex. \\

Though we achieve a relatively good performance from a LBM model on SpiNNaker. Multicast may not be a perfect match for LBM. In lattice Boltzmann method, a lattice only need one distribution function from a neighbour in one direction, where a point-to-point communication is more suitable.\\

Lattice Boltzmann method is a relatively simple model compared with other CFD models and it has been proven that SpiNNaker is capable of LBM simulation. However, the ITCM in SpiNNaker core is limited, only 32kB, and when porting a more complex CFD model to the SpiNNaker platform, the size of ITCM would be a hindrance. \\

 



\subsection{Future Work} \label{sec:future_work}
SpiNNaker is a good platform for running CFD simulation. In this project, we have not yet engaged its full capacity. There are basically two ways to go further.\\

The first one is on the CFD side, in this project, we only implement the most basic case in LBM. There are different boundary conditions, optimization method yet to implement. Besides, there are still many CFD models that can benefit from SpiNNaker's massive parallelism. Therefore, in the future, more CFD models can be ported to SpiNNaker for performance. \\

The second one is on the SpiNNaker side, in this project, we only allocate one lattice per core. Therefore, the simulation scale is very limited. In theory, a SpiNNaker core is able to run over 256 lattices and in the near future this number would be over 1,000. In this case, the SpiNNaker will benefit to with real-world engineering problems.\\




