\documentclass[12pt, twoside]{article}

\title{A Lattice Boltzmann Method \\ On SpiNNaker }
\author{Yuan Feng (UUN: s1909558)}

% include files that load packages and define macros
\input{includes/includes} % various packages needed for maths etc.
\input{includes/notation} % short-hand notation and macros

\setlength{\parindent}{5ex}

\begin{document}
\input{title/title.tex}

\newpage
\begin{abstract}


\noindent Computational fluid dynamics has a wide range of applications in areas such as Formula 1 racing. However, while some fluid dynamics models such as the lattice Boltzmann method are recognized as easy to parallel, existing parallel hardware techniques receive continued challenges as the requirements of engineers and scientists for CFD simulations increase further. SpiNNaker, a bio-inspired hardware has advantages such as massive parallelism and low power consumption, bring opportunities to CFD simulation. In this project, we take the hardware specificity as an opportunity by porting a CFD algorithm, the lattice Boltzmann method, to the SpiNNaker platform to investigate the scaling ability and compare it with a standard serial implementation on normal CPU. To do this, we: (a) implement a basic lattice Boltzmann method on SpiNNaker platform with over 60,000 cores; (b) demonstrated that the lattice Boltzmann method on the SpiNNaker platform has good performance of speed and advantage in weak-scaling.

\end{abstract}



\newpage
\renewcommand{\abstractname}{Acknowledgements}
\begin{abstract}

\noindent Sincerely thanks go to my supervisors: Dr Kevin Stratford and Dr Alan Stokes. This project will not be impossible without their continuous guidance. Thank you to the EPCC people for the online supporting. Thank you to the APT group for building the SpiNNaker. Thank you also to the SpiNNaker User Group. Without the wonderful people from the User Group, it would be even harder during lock-down. Finally, special thanks go to Yuke Li for backing me all the way to the end.

\end{abstract}


\newpage
\tableofcontents
%\listoffigures
%\listoftables

\newpage
\section{Introduction}

\subsection{Motivation}


Fluid is everywhere, from the air we breath, the water we drink to the wind flow around the air craft. Computational fluid dynamics (CFD) is a science that \cite{thelbmbible}, with the help of digital computers, produces quantitative predictions of fluid-flow phenomena based on the conservation laws (conservation of mass, momentum, and energy) governing fluid motion. Computational fluid dynamics has been applied to many aspects of our society, including wind simulation for aircraft, fire simulation in environment engineering and even the airflow around face masks simulation during the COVID-19. On the commercial side, the market size of CFD industry in 2016 is over 20 times larger than 2001; see Fig.~\ref{fig:cfd_market}. In 2019, The Worldwide Computational Fluid Dynamics (CFD) market size was USD 1703.5 million and it is expected to reach USD 3200.3 million by the end of 2026\cite{market_cfd}.\\

\begin{figure}[htbp]
    \centering
    \includegraphics[width=1\textwidth]{figures/CFD_market.jpg}
    \caption{The global market size of CFD industry. From the early 1983, the market size has been increasing significantly reaching over 1200 million dollars by 2014. The trend has been predicted to continue. (Image from \cite{market_cfd})}
    \label{fig:cfd_market}
\end{figure}

As the definition says, CFD highly rely on computers, especially high-performance computers (HPC). Indeed, quite an objective part of HPC resource also goes to fluid dynamics simulation; see \ref{fig:break_down}. However, the simulation is becoming more demanding, and the CFD poses new challenges to HPC every year. For example, Formula 1 racing teams give high priority to aerodynamics simulation and they always want to make as much simulation as possible within a short period of time. If they want to beat the other F-1 teams in the aerodynamics, they need to run simulation faster in a more powerful supercomputer while the Fédération Internationale de l'Automobile (FIA) start to limit the power consumption of the CFD simulation \cite{formula1} of F-1 teams. Engineers like them from all areas face the similar challenge and they need to figure out a feasible way.

\begin{figure}[!tb]
    \centering
    \includegraphics[width=0.7\textwidth]{figures/break_down.png}
    \caption{The HPC usage in ARCHER by research area in July 2016. Computational fluid dynamics related areas (including mesoscale simulation, combustion modeling and CFD) took over 10\% of the total use. (Image from \cite{archer_use}))}
    \label{fig:break_down}
\end{figure}


% \begin{figure}[!tb]
%   \centering
%       \includegraphics[width=0.7\textwidth]{figures/cluster.png}
%       \caption{The 500,000-core SpiNNaker \textit{102} machine at the University of Manchester (image from \cite{spinn-core}).}
%       \label{fig:cluster}
% \end{figure}


% Nowadays, the amount of data has grown increasing large, traditional computer architectures do not scale as the Moore's law. While parallelism is regarded as an option to keep the scaling, the SpiNNaker hardware was designed to be massively distributed and parallel. Thus the SpiNNaker hardware has the potential to be used as a parallel computing solution. As opposite to the traditional hardware, SpiNNaker has the following hardware features:

% \begin{itemize}
% \item \textbf{Manycore architecture}: though modern CPUs start to have multiple cores to gain some parallelism, the number of cores in a CPU is usually less than 10. On the opposite, a type Spinn-5 SpiNNaker board has 48 chips, which contains hundreds of ARM cores \cite{5th-summit}. Though there are more cores, because of the medium-performance (only 200MHz) ARM968 cores, the energy consumption stays relatively low.

% \item \textbf{Communication model}: the SpiNNaker cores communicate by sending message via UDP/IP \cite{ws6} and UDP/IP do not guarantee the delivery. Though it is the developers' responsibility to make sure the correctness, the developers do not need to worry about the dead-lock. It is designed as so because in a real human brain a neuron does not get any an acknowledgement when the communication is done \cite{spinnaker}.

% \item \textbf{Communication throughput}: extra performance might be gain from the reduce of the message size and more message can be process in the same amount of time\cite{furber2012overview}. Their link can transfer up to 31.25M byte/s, which means the links can handle 3M packets per second \cite{ws6}.\\
% \end{itemize}

Meanwhile, some neuromorphic hardware has been actively developed during the same period of time. SpiNNaker is one of them. SpiNNaker (Spiking Neural Network Architecture) \cite{thespinnbible} is an approach to build a machine that is based to some degree on what is understood about the principles of operation of the human brain. It has the following hardware features:
\begin{itemize} 
\item \textbf{Massively parallel:} A single type Spinn-5 SpiNNaker board has 48 SpiNNaker chips. Each SpiNNaker chip has 18 ARM cores, which means the board has up to $48 \times 18 = 864$ cores. Each core has the ability to simulate at least 256 neurons. The SpiNNaker clusters are build with cabinets of boards. The SpiNNaker can bring massive parallelism.

\item \textbf{Low energy-consumption:} Opposite to the conventional HPC which are usually build up with energy-hungry cores running in GHz, the SpiNNaker only need 1 watt per chip.

\item \textbf{Communication model:} The communication model we primarily used is multicast. The cores can pass messages to multiple destinations by a single send.
\end{itemize}


In a word, the SpiNNaker has the full features as a supercomputer and it could provide new opportunities for parallel software engineers and CFD engineers. However, there is not much scientific applications developed on SpiNNaker especially CFD applications. In this project, we will explore the potential of using SpiNNaker for CFD simulation and trying to achieve a good performance in terms of speed.\\
 
% On the other side, Computational fluid dynamics (CFD) is a science that, with the help of digital computers, produces quantitative predictions of fluid-flow phenomena based on the conservation laws (conservation of mass, momentum, and energy) governing fluid motion \cite{thelbmbible}. It help us to to solve real-world engineering problems, including aerospace engineering, meteorology, etc. However, CFD algorithm usually involve heavy computation. To get the simulation result in affordable time, scientists and engineers need to accelerate the simulation. Therefore, parallel computers including some supercomputers are heavily used for CFD simulation.\\

Lattice Boltzmann method (referred to as LBM in the rest of this repory) is a mesoscopic CFD model which is recognized\cite{lbmmbook} as: (1) easy to apply to complex domain (2) No need to solve the Laplace equation (3) More importantly for this project, being naturally adapted to parallel processing due to the locality and explicit nature of the method. Due to its (3) nature, many parallel techniques are applied to accelerate the LBM simulation, including MPI\cite{he1999three}, OpenMP\cite{massaioli2002achieving} and GPGPU\cite{rogers1990upwind}, etc. \\

As we discussed above, the SpiNNaker has the ability to compute in massively parallel. Thus, there is a perfect match between the SpiNNaker and the lattice Boltzmann method. It is promising that we can get high performance and scalability in term of speed with SpiNNaker's low energy-consumption cores on the LBM simulation tasks, which is also the motivation behind this project.  \\




\subsection{Objectives} \label{sec:Obj}

Firstly, we can define our objective of this projects as follow:\\

\begin{quote}
Implementing a basic lattice Boltzmann method simulation on the SpiNNaker platform; and investigate its  performance and scalability in terms of speed. \\
\end{quote}

% KS: In point one there are two separate points which should be discussed seaparately. First, there is the implementation of the algirthm. Second, there
% is the use of a standard test problem to check that the implementation is
% working correctly. A wide range of test problems could be used.

% KS: There needs to be more here. It's not just a question of implementation. There
% needs to be some critical evaluation of ease-of-use, performance, and so on. 
% to add to KS. you need to discuss how your going to compare. speed, scale? energy? accuracy?

Firstly, we need to implement a standard lattice Boltzmann method on CPU as a reference, which can be more easily to understand and porting the algorithm; and we can also further evaluate the correctness, accuracy and compare the performance with this CPU implementation.\\

Secondly, we need to choose a pre-defined problem as the test problem from a wide range of the problems. As the test problem, it need to be generic and easy to check the correctness. To choose it, we take LBM model and the boundary condition into consideration, and, finally, a two dimensions and nine vectors (D2Q9) model with a periodic condition problem described by Minion and Brown \cite{minion1997performance} was chosen. It is generic -- D2Q9 model is widely used, and easy to check the correctness -- with the mentioned initial condition, there will be a turbulence in a fix step and we can check it.\\

Then, we can focus on the design and implementation of the LBM on the SpiNNaker.

Finally, after implementing the simulation on SpiNNaker, we will to evaluate correctness and accuracy the simulation result by compare the numbers in quantitatively. After we confirm the correctness, them some experiment would be focus on optimization the communication and bench mark the result with the standard CPU implementation on speed-ups and scalability.\\

\subsection{Project Overview}

The work of this project are threefold:\\

 \textbf{A basic lattice Boltzmann implementation on CPU}: we firstly built a standard serial implementation of a pre-defined lattice Boltzmann scenario described by Minion and Brown \cite{minion1997performance}; see Subsection \ref{sec:ip}. \\

 \textbf{A basic lattice Boltzmann implementation on SpiNNaker:} after we implemented the simulation on CPU, we used the CPU implementation as a reference to implement the same simulation on SpiNNaker platform with some of the SpiNNaker software development kit; see Section \ref{sec:dai}\\

\textbf{An investigation on the speed performance and scalability}: we demonstrate that lattice Boltzmann on SpiNNaker platform offers better speed performance in some large scale when compared to CPU and it also has a good scalability on weak-scaling. We bench-marked two implementations of the lattice Boltzmann method scenario: the first one is the standard implementation of on a normal Intel CPU and second one is a implementation on SpiNNaker mentioned above. The observation shows that lattice Boltzmann program can gain 5x speed-up in terms of speed over a certain scale; see Section \ref{sec:eval}.




\newpage
\section{Background} \label{sec:bg}


This section will provide a basic background information of SpiNNaker including an overall comparison between SpiNNaker and CPU in HPC(\ref{sec:sa}), hardware architecture (\ref{sec:sa}) and SpiNNaker software stack (\ref{sec:sss}). Then this section will explain some important conceptions in physics (\ref{sec:PB}) about the lattice Boltzmann method: why it is necessary and how it works. 


% \subsection{History in a Nutshell} \label{sec:sb}
% The vision of \textbf{neuromorphic computing}\cite{mead1980introduction} is to enable a new generation of computer architecture, designing energy-efficient general-purpose computing systems comparable to the human brain. In 1989, Carver Mead from Caltech first introduced the concept of \textit{neuromorphic Engineering} \cite{mead1980introduction}. From 1990 to 2003, the Von Neumann-based CPU industry continued to grow, Moore's Law \cite{schaller1997moore} was reaching its limits, and neuromorphic computing was dormant for more than a decade. In around 2004, the frequency growth of single-core processors slowed down, and IC designers turned to multi-core processors. Academia began to look for alternative technologies to the Von Neumann architecture.\\

% In 2004, Kwabena Boahen from Stanford University developed Neurogrid \cite{benjamin2014neurogrid}, an analog circuit-based neural chip. In 2005, the University of Manchester began research on asynchronized communication, which is the precursor of SpiNNaker. In the same year, the Europe, the US and IBM started their neuromorphic computing projects: FACETS project \cite{meier2004fast}, SyNAPSE project\cite{park2014impact} and Blue Brain project\cite{gara2005overview}, respectively. \\

% In 2013, A 10-year-project, \textbf{Human Brain Project} began in 2013. It aims at building a research infrastructure to help advanced neuroscience, medicine, and computing \cite{hbp}. Although, the SpiNNaker project had started even before the Human Brain Project funded by the \textit{Engineering and Physical Sciences Research Council} (EPSRC), the Human Brain Project bring more funding and brilliant researchers building the 1 million core SpiNNaker machine since November 2018; see Fig~.\ref{fig:super_machine}.

% \begin{figure}[!tb]
%   \centering
%       \includegraphics[width=0.8\textwidth]{figures/super_machine.jpg}
%       \caption{Since November 2018: The 1 million core SpiNNaker HBP Platform machine((image from \cite{super_machine})).}
%       \label{fig:super_machine}
% \end{figure}

% \begin{figure}[htbp]
%   \centering
%       \includegraphics[width=0.8\textwidth]{figures/spinn_labeled_bw.png}
%       \caption{An 18 cores SpiNNaker chip ((image from \cite{spinn-core})).}
%       \label{fig:spinn-core}
% \end{figure}

\subsection{An Overall Comparison} \label{sec:sb}
SpiNNaker is a novel hardware architecture that people rarely have opportunity to access it. To give an overall idea of SpiNNaker for programmers who are more familiar with multi-processor programming with CPU, especially MPI in HPC field. We will compare them briefly as following:

\begin{itemize}
    \item \textbf{Operating System:} While most HPC facilities have full-feature operating system (mostly Linux), SpiNNaker works as connected device (like GPU). 
    
    \item \textbf{Parallelism:} Top HPC facilities, such as ARCHER, would have over 100,000 cores, while SpiNNaker cluster in the University of Manchester have over 1,000,000 cores which might expose more parallelism to parallel application developers.
    
    \item \textbf{Energy consuming:} While HPC facilities tend to use energy-hungry CPU in GHz, SpiNNaker machines use medium-performance cores. As a consequence, a SpiNNaker consumes one watt per chip.
    
    \item \textbf{Development:} In MPI, developers compile their C code with MPI compile command (mpicc) then run the executable with MPI run  command (mpirun). In SpiNNaker, developers write C code to define the behaviors (computation and communication) for each individual core and write Python scripts to define communication pattern and data specification.
\end{itemize}


\subsection{SpiNNaker Architecture} \label{sec:sa}
In this subsection, we will illustrate the basic architecture of the SpiNNaker.
\subsubsection{SpiNNaker Core} \label{sec:ca}

Fig.~\ref{fig:spinn-core} is an diagram of ARM968 core used in SpiNNaker hardware. Being different from the most cores in the PCs or supercomputer running at $\sim GHz$, ARM968 is a medium-performance computing unit running at 200MHz with 220 DMIPS \cite{furber2012overview}. There are two very limited tightly coupled memory (TCM) blocks: a 64 Kbytes of data tight-coupled memory (DTCM) block and a 32 Kbytes of instruction tight-coupled memory (ITCM) block; see Fig.~\ref{fig:arm_968}. Other controllers including direct memory access (DMA) controller, communication controller, vectored interrupt controller are also built in. A major difference from the most cores in the market is that the ARM968 core do not have floating-point hardware and use fixed-point arithmetic instead. Though it might be more energy-efficient, it bring greater challenge to programmers \cite{furber2012overview}. Luckily, a software floating-point arithmetic is supported, though the performance will be slower and more memory-hungry \cite{spin-chip-resources}.
    \begin{figure}[!tb]
   \centering
       \includegraphics[width=1\textwidth]{figures/core.png}
       \caption{A diagram of SpiNNaker Core -- ARM 968 ((image from \cite{spin-chip-resources})).}
       \label{fig:arm_968}
    \end{figure}

\subsubsection{Network Topology}
Before we talk about how the SpiNNaker cores communicate, we need to illustrate how they are connected with its network topology.

In a SpiNNaker board, SpiNNaker chips are organized as a 2-dimension mesh network with bidirectional links to their six neighbours \cite{testchip}; see Fig.~\ref{fig:topology}. The cores sit on the border are connected with cores on the other end of the board.

\begin{figure}[!tb]
   \centering
       \includegraphics[width=0.8\textwidth]{figures/topology.png}
       \caption{The topology of nine SpiNNaker cores. Each core is connect with six other cores via bidirectional links. The cores on the border are connected with the core on the other end periodically.}
       \label{fig:topology}
\end{figure}

\subsubsection{Data Transmission in SpiNNaker} \label{sec:dt}
Though there are a few different kinds of packets that SpiNNaker can use for communication, e.g. Point-to-point packet, those packets are not primarily for general purpose development on SpiNNaker. In this project, in fact for most SpiNNaker software development, the multicast packets would be used.\\

As the name says, the multicast packets are sent via SpiNNaker multicast; and with this communication pattern, the packets are sent once to multiple destinations along with the routing key of the senders; see Fig.~\ref{fig:multicast} The routing key is allocated by the SpiNNaker system runtime. \\

It should be pointed out that in actual development the payload of the multicast packet should be in the format of 32bit unsigned integer. Developers need to either send 32bit unsigned integer or convert the payload into this format.\\

\begin{figure}[tb]
\centering
\includegraphics[width = 0.7\hsize]{figures/multicast.png}
\caption{A Multicast show case, where core (0,0) send packet once to multiple destinations, core(1,1), core(1,2), core(0,1) and core(2,2).(image from \cite{ws6}).}
\label{fig:multicast}
\end{figure}

As we discussed before, \textbf{the multicast communication in SpiNNaker do not guarantee the delivery}, and as a consequence, the packets might be drop at any time. Luckily, the re-injector will re-inject the dropped packets if there is any. However, it is still possible to drop packets during communication. When there is intensive message-passing, with a high peak packets rate, there would be back pressure and packet collision. We will discuss the difficulty and solution at Section~\ref{sec:co}.\\


\subsection{SpiNNaker Software Stack } \label{sec:sss}
    \begin{figure}[!tb]
        \centering
       \includegraphics[width=1\textwidth]{figures/software_stack.png}
       \caption{A diagram of SpiNNaker Software Stack ((image from \cite{spin-chip-resources})).}
       \label{fig:software_stack}
    \end{figure}

Fig.~\ref{fig:software_stack} shows what SpiNNaker software \cite{software_spinn} contains and how it works. High-level abstractions for Neural simulation (\textit{sPyNNaker}) and general parallel computing (\textit{SpiNNaker GraphFrontEnd}) provide APIs for developers to describe the neural connection or computational graph. Since we are not developing neural simulation application, the primary software we used for this project is the SpiNNaker GraphFrontEnd. \\

After the computational graph is described by the developers in Python, the Partition and Configuration Manager system (\textit{PACMAN}) will map the computational graph to the SpiNNaker cores. \\

The actual communication of each vertex is defined via the \textit{Spin1} API in C code. The \textit{Spin1} (short for SpiNNaker1 API) is a set of low-level, on-chip libraries implementing the SpiNNaker's event-based system. The programming model of SpiNNaker is event-driven model, where message from outside would trigger some behaviour (called a \textbf{callback}) of the SpiNNaker vertices. The SpiNNaker real-time OS schedules threads to response to callbacks in a given priority.\\

Sitting on the lowest layer of the software stack, the SpiNNaker application runtime kernel (\textit{SARK}) support the runtime system of the SpiNNaker core. We will not touch software in such low layer in this project.\\


\subsection{Development Environment for SpiNNaker} \label{sec:impl}
There are many ways to access the SpiNNaker hardware. In this project, two methods were applied. This section will briefly discuss the advantages and disadvantages of those two method and why we changed the access method.\\

\subsubsection{Directed Connected Board with IDE}
The first choice for this project is to run the simulation with physically board connected with the laptop. As shown in the Fig.~\ref{fig:laptop}, with a physically board, SpiNNaker developers can make all the development without the internet. Typically, they write code in their host computer with a code editor or an IDE then offload the simulation to the connected SpiNNaker board.\\

One of the advantages of development with a SpiNNaker Board is that the developers can physically watch the running condition of the SpiNNaker boards and make corresponding manipulation. Another advantage is that the developers can make full use of their favourite code editors or IDEs, which makes it easier to view the source code and debugging.\\

There are some disadvantages though. A major disadvantage is that it is the developers' duty to maintain the development condition of the SpiNNaker board including the Ethernet connection, electronic power and the condition of the SpiNNaker board itself. Most SpiNNaker users are not professional electronic engineers, so if there is any trouble with those problem, the developers must try to get help from the SpiNNaker team or they are on their own. \\

\begin{figure}
\centering
   \centering
       \includegraphics[width=0.8\textwidth]{figures/laptop.png}
       \caption{Development with Directed Connected Board via Ethernet switch ( board image from \cite{spinn-core}). Developers need a physical board, but they can use their desirable development environment.}
       \label{fig:laptop}
\end{figure}


\subsubsection{Web interface with Jupyter Notebook}
For most SpiNNaker users or who are just get into the SpiNNaker development, they may not even have a board. The best way to write SpiNNaker to develop SpiNNaker application is using the web interface development by the SpiNNaker team. \\

With the web interface, the developers run the python code within the Jupyter Notebook. The Jupyter also provides a terminal enabling the developers compile their C code and run shell command. For this project, we compile the C code on the laptop and upload the binaries to the remote system and run the application in the Jupyter notebook(shown in Fig.~\ref{fig:jupyter}).\\

The biggest advantage of developing with the Jupyter interface is that the developers do not need to take extra concern on the board maintenance. They can pay full attention on the application development even if they do not even have a board. Another advantage is that there are more than 1 million cores available remotely. They developer can scale their application up with the Jupyter notebook easily. Though it might be easier for developer to connect with a SpiNNaker machine via the Jupyter, it is hard to view the source code of the APIs and debugging in the Jupyter. \\

Overall, for hardcore developers or absolute beginners, it might be better to use the connected board to develop SpiNNaker framework or applications and getting familiar with the SpiNNaker API stack. For those who have been already familiar with the SpiNNaker API stack and want to scale their application up, it might be better to use the Jupyter interface.\\

At the beginning of this project, our first choice to use a Spinn-5 SpiNNaker machine directly connected with a personal computer via Ethernet. However, later on, there was a problem with the connection port. Due to the pandemic COVID-19, the repair is not available. We then switch to the web interface with the Jupyter Notebook. There are many ways to get access to the SpiNNaker machine, we will mainly discuss those two method that are used during this project.\\

\begin{figure}[!tb]
\centering
    \centering
   \includegraphics[width=0.8\textwidth]{figures/jupyter.png}
       \caption{Development with Jupyter Web interface ( board image from \cite{spinn-core}). The SpiNNaker team has maintained the environment and user can development directly in Notebook, but with less customized environment and less convenient debugging experience. If the developer has the access to the University of Manchester account, they have more options that are even more convenient. But it do not apply to this project.}
       \label{fig:jupyter}
\end{figure}

\subsubsection{SpiNNaker Development Workflow} \label{sec:sdw}

For this project, instead of neural simulation, we are using the SpiNNaker as a parallel computer and our workflow of development are based on the GraphFrontEnd API. In the development, as we discussed above, the SpiNNaker Jupyter Notebook interface were used after some early development on personal computer. The SpiNNaker project is developing fast and the software iterates pretty quickly. Since we have some early development with a relatively stable version of software, though the Jupyter interface do provide a terminal for compile and build, to keep the continuity of development, we kept using personal computer for C source code compiling. After we get the binary files, we then upload them to the Jupyter interface. \\

At the same time, we need to write Python scripts to define the SpiNNaker vertices and edges via the SpiNNaker \textbf{PACMAN} API. The vertices contain computational entities and the edges define the connection information between vertices; and the \textbf{PACMAN} library would map the vertices to the SpiNNaker machine.\\

Eventually, another Python script should be provided as an entrance to run the simulation and get data back to the host. This script is based on the SpiNNaker \textbf{GraphFrontEnd} library. This library would provided APIs for developers to control the configuration of SpiNNaker simulation.\\



\subsection{Physics Background} \label{sec:PB}
% KS: Need to clearly separate what is the kinetic theory of gases and the Boltzmann
% equation, and what is lattice Boltzmann. REFERENCES?

% KS: What are the Navier Stokes equations?
\subsubsection{Fluid Model in Brief}
Fluids are physically discrete systems composed of a large number ($~10^{23}$) of particles, where every particle is constantly doing Brownian motion and exchange momentum and energy by collision. Thus the microscopic structure and motion are quite complex in both space and time. On the other hand, contrary to the inhomogeneity, dispersion, and randomness of microscopic motion, the macroscopic motion of the fluid exhibits uniformity, continuity, and determinism. The macroscopic motion and other properties of the fluid are the result of averaging the microscopic motions of the fluid molecules. Therefore, the mathematical models describing fluid motion can vary considerably when observed at different scales.\\

In general, methods for describing fluid systems can be classified into Molecular Dynamics model, Mesoscopic Model and macroscopic continuum models depending on the scale; see Fig.~\ref{fig:fluid_models}. The Molecular Dynamics model views the fluid as a many-body system consisting of a large number of molecules and focuses on the dynamic behavior of each fluid molecule (at \ref{sec:HE}). Through the movement of each molecule of the statistical representation to describe the overall motion of the fluid; macroscopic continuum model of the fluid as a continuous whole, focusing on the fluid microscopic group, with a set of partial differential equations (Navier Stokes Equations \ref{sec:nse}) to describe the macroscopic motion of the fluid; mesoscopic dynamic model, including the lattice Boltzmann model, focuses on the velocity distribution function of the fluid molecules, by expressing its macroscopic physical quantities and distribution function over time to obtain macroscopic flow Information (Boltzmann Equation \ref{sec:BE}). \\

\begin{figure}[!tb]
   \centering
       \includegraphics[width=1\textwidth]{figures/comparsion.png}
       \caption{Three different kinds of fluid models.The one on the top represent macroscopic models, including finite difference, finite element, etc, which mainly apply the Navier-Stokes Equations. The one in the middle represent mesoscopic models, such as LBM. Those models apply Boltzmann Equation. The bottom one represent microscopic models, which apply Hamilton's mechanics/equations. }
       \label{fig:fluid_models}
\end{figure}

The lattice Boltzmann method (LBM) as a mesoscopic model set in the between the microscopic model and the continuum model. It enjoy the advantages of both the macroscopic and microscopic approaches, and we will introduce both of them next before the LBM.

\subsubsection{Navier-Stokes Equations -- Macroscopic Model} \label{sec:nse}
In general, a fluid can be viewed as a continuous medium that fills the entire flow field, so that the physical quantities of density, velocity, temperature, etc. can be defined at each point of the flow field and a series of partial differential equations can be established to describe the motion of the fluid. The continuous medium assumption is a fundamental assumption of fluid mechanics and is an approximate treatment of macroscopic fluid structure.\\

Based on the assumption of a continuous medium, the motion of a fluid follows the law of conservation of mass, momentum and energy. In this project, we do not need to consider the temperature (T), energy (E) or heat flux (q). In this case, the system need to meet the conservation of mass and momentum; see Equation~\ref{equ:conservation} and $f$ is the force on the fluid. 

% \begin{equation}
%     \left\{\begin{matrix}
% \frac{\partial \rho}{\partial t} + \triangledown \cdot (\rho u) = 0\\ 
% \\
% \frac{\partial (\rho u)}{\partial t} + \triangledown \cdot (\rho u u) = 0\\ 
% \\
% \frac{\partial (\rho u)}{\partial (t) + \triangledown \cdot (\rho u e)} = \sigma : \triangledown u - \triangledown \cdot q
% \end{matrix}\right.
% \label{equ:NS}
% \end{equation}

% Here in Equation ~\ref{equ:NS}, $\rho$, $u$, $T$ and $e$ are the density, speed, temperature and internal energy per unit mass, respectively. $\sigma$ is the strain tensor and $q$ is the heat flux from heat transfer and heat radiation.

% The Equation ~\ref{equ:NS} is not in closed-form. In order to obtain the complete equations, the equations of states need to be supplemented with the relationship between the stress tensor and the rate of deformation tensor, between the heat flow vector and the temperature gradient, and with the thermodynamic properties of the associated thermodynamic properties. For Newtonian fluids, the stress tensor is linearly related to the deformation rate tensor and can be expressed as, where $I$ is the second-order unit tensor, $p$ is the static pressure, $\tau = 2 \mu S + \lambda(\triangledown \cdot u)I$ is the viscous stress tensor, where $\mu$ is the dynamic viscosity coefficient, $\lambda$ is the second viscosity coefficient, $S$ is the deformation coefficient tensor, and $a$ and $b$ is the index of the tensor which is defined as Equation~\ref{equ:DCT}:

% \begin{equation}
%     \label{equ:DCT}
%         S_{a b} = \frac{1}{2} (\frac{\partial u_a}{\partial x_b} + \frac{\partial u_b}{\partial x_a})
% \end{equation}

% Under the Stokes' hypothesis \cite{gad1995stokes}, the two types of viscosity coefficients are related by: $\lambda + (2/3)\mu = 0$. By applying the Stokes' hypothesis, the Equation ~\ref{equ:NS} can be called Navier-Stokes equations \cite{lbmmbook}. \\



\begin{equation}
    \label{equ:conservation}
        \begin{matrix}
        \frac{\partial \rho}{ \partial t} + \nabla (\rho u) = 0 \\
        \frac{\partial{\rho u}}{\partial t} + \nabla \cdot (\rho u u ) = f \\
        \end{matrix}
\end{equation}

The external force $f$ can be written as Equation~\ref{equ:force}, where $p$ in the isotropic pressure and $\tau$ is a stress.
\begin{equation}
    \label{equ:force}
    f = -\nabla p + \nabla \cdot \tau
\end{equation}

Correspondingly, according to the Stokes assumption, the stress $\tau$ can be written as Equation~\ref{equ:stks}, where $\eta$ is the dynamic viscosity which describes, at a macroscopic scale, the effect of collisional diffusion of momentum between molecules. 
\begin{equation}
    \label{equ:stks}
    \tau_{ab} = \eta (\frac{\partial u_a}{ \partial x_b} + \frac{\partial u_b}{\partial x_b})
\end{equation}


\subsubsection{Molecular Dynamics model -- Microscopic Model} \label{sec:HE}
The continuum models with Navier-Stokes equations do not take into account the microscopic molecular model of the fluid and directly describes the macroscopic physical quantities of the fluid, but the fluid is physically composed of fluid molecules, and the macroscopic motion of the fluid is the result of averaging the thermal motion of the microscopic molecules. Therefore, if the microscopic motion of the fluid molecules can be known, the macroscopic physical quantities of the fluid can theoretically be obtained by averaging. This is the basic idea of the molecular dynamics model. The molecular dynamics model looks at the microscopic molecular motion of a fluid, studies the time evolution of the spatial position and velocity of the fluid molecules, etc., and uses statistical methods to obtain information on the macroscopic flow from the microscopic information of the molecules.\\

In general, Hamiltonian equations are applied to this model \cite{salmon1988hamiltonian}, but the reasoning is beyond the scope of this project. With the idea of describing the macroscopic physical quantities by averaging the microscopic particles, we can then move to the kinetic theory of gas.

\subsubsection{Kinetic Theory of Gas}\label{sec:BE}
Kinetic theory is the branch of statistical physics dealing with the dynamics of non-equilibrium process and their relaxation to thermodynamic equilibrium \cite{succi2001lattice}. The theory of gas kinetics is based on the molecular model, which holds that a gas is made up of a large number of molecules ($10^{23}$) and that the molecules are always in constant random motion. The interaction between any two molecules can be expressed as a function of distance, such as the simplest molecular model of a rigid sphere model, which holds that two molecules interact only when they are in contact, and the inter-molecular interaction function is Equation~\ref{equ:hard_sphere}:\\

\begin{equation}
\label{equ:hard_sphere}
\phi  (r) = \left\{\begin{matrix}
\infty, & r \leqslant \sigma \\ 
0, & r > \sigma
\end{matrix}\right.
\end{equation}

Here, $r$ is the distance between two molecular and $\sigma$ is the diameter of molecular.\\

However, for gas systems consisting of a number of molecules ($10^{23}$), tracking the motion of each molecule based on the forces between molecules as in molecular dynamics is impractical for most systems. Applying statistical methods to study the statistical characteristics of these discrete molecules, such as the average number of molecules in a small volume unit and over small time intervals, the average velocity, the average energy, and other relevant physical quantities, is a feasible approach, and this is the fundamental point for kinetic theory of gas and the Boltzmann Equation.

\subsubsection{Boltzmann Equation -- Mesoscopic Model}
Ludwig Eduard Boltzmann (1844-1906), the Austrian physicist explains and predicts how the properties of atoms and molecules (microscopic properties) determine the phenomenological (macroscopic) properties of mater including viscosity, thermal conductivity\cite{lbmmbook}. He proposed and developed the Boltzmann equation, which then become the starting point of the kinetic theory of gas.\\

To describe the Boltzmann equation, in any macroscopic system, the microscopic motion of each molecule follows the laws of mechanics, so as long as the individual motion of a large number of particles, you can determine the macroscopic parameters of the whole system, which is the basic starting point of molecular dynamics; from another perspective, instead of determining the state of motion of each molecule, we can find the probability of each molecule in a certain state, which can be obtained by statistical methods. This is the basic idea behind the Boltzmann equation, which is the equation used in statistical mechanics to express the evolution of an non-equilibrium distributed function; see Equation ~\ref{equ:BE}.

\begin{equation}
\label{equ:BE}
    \frac {\partial f}{\partial t} + \vec u \cdot \triangledown _x f + \vec a \cdot \triangle _ u f = \Omega (f)
\end{equation}

In Equation ~\ref{equ:BE}, $f$ is the velocity distribution function; $u$($u_x$,$u_y$,$u_z$,) is the the molecular velocity vector; $t$ is the time; $a$ is the acceleration (given by the external force $F=ma$); $\Omega$ is the collision operator or collision term \cite{succi2001lattice}.\\

The Boltzmann equation (Equation~\ref{equ:BE}) is a complex integro–differential equation. It is unrealistic to get an exact solution. In this case, lattice method become a feasible way to compute and model the fluid. Lattice Gas Automata (LGA) \cite{frisch1986lattice} is one of them; and the lattice Boltzmann method derived from it.\\

But before introduce the lattice Boltzmann method, we need to talk about another important concept, Bhatnagar-Gross-Krook Collision (\ref{sec:BKG}).\\

\subsubsection{Bhatnagar-Gross-Krook approximation} \label{sec:BKG}
Because of the close relationship between the Boltzmann equation and the fundamental equations of fluid mechanics, the Boltzmann equation could be solved numerically to simulate the macroscopic motion of a fluid. However, since it is not practical to solve Boltzmann equation directly, the biggest difficulty lies in its collision term. Therefore, it is a natural idea to use a simple form of collision instead of the collision term, and the Bhatnagar-Gross-Krook (BGK) approximation/collision \cite{bgk} arises in this context.\\

BGK approximation was firstly proposed by Bhatnagar, Gross and Krook in 1954 \cite{bgk}. They think that a collision term should: 

\begin{itemize}
\item \textbf{Satisfy the conservation of mass, momentum and energy}. 

\item \textbf{Be able to reflect the tendency of the system towards equilibrium}
\end{itemize}

A simple collision term can draw from those two assumption with assuming the effect of a collision is to change the distribution function $f$ so that it tends to an equilibrium distribution $f^{eq}$. Set the rate of change to be proportional to the difference between $f$ and $f^{eq}$, and $\tau$ is the relaxation factor (reverse time). So you can introduce a BGK collision term $\Omega (f)$ at Equation ~\ref{equ:BGKLB}:

\begin{equation}
\label{equ:BGKLB}
    \Omega (f) = \frac{(f_{eq} - f)}{\tau}
\end{equation}

Equation~\ref{equ:BGKLB} is called Boltzmann\-BGK equation \cite{chew1956boltzmann}. The BGK approximation greatly simplifies the solution of the equation.\\

\subsubsection{Lattice Boltzmann Method in Brief} \label{sec:lbmb}
The basic idea of the lattice Boltzmann is imagine the fluids including gases as a great number of particles that are moving with random states. The particles exchange their momentum and energy by streaming and collision. We can describe the process as the Boltzmann transport equation Equation ~\ref{equ:BTE}:
\begin{equation}
\label{equ:BTE}
    \frac{\partial f}{\partial t} + \vec{u}\cdot \nabla f = \Omega
\end{equation}
In Equation~\ref{equ:BTE}, $f(\vec{x}, t)$ is the distribution function, $\vec{u}$ is the velocity of particles and $\Omega$ represent the collision term. In this project, the lattice would be simplified to be in two dimensions. The velocities $\vec{u}$ are described as \textit{macroscopic velocities}, whilst in this project the there would be 9 \textit{microscopic velocities} (seen in Fig.~\ref{fig:d2q9} and labelled as $\vec{e}_1...\vec{e}_9$). \\

In 1992, Qian et al. \cite{d2q9} proposed DdQm ($d$ dimensions, $m$ discrete velocities) models, which are the basic models of LBM. Fig.~\ref{fig:d2q9} shows a D2Q9 model and its nine discrete velocities in two dimensions. Correspondingly, in Equation.~\ref{equ:d2q9}, we introduce the discrete velocities $e_i$ in two dimensions.\\

% KS: I would try to put Figures either at the top or the bottom
% of the page so they don't interupt the text, so use tb in the
% command for figure. Figures should be 'called out' in the text
% so put a label in the figure and reference it in the text.


\begin{figure}[!tb]
   \centering
       \includegraphics[width=0.5\textwidth]{figures/nine_direction.jpg}
       \caption{A lattice in D2Q9 model and its nine discrete velocities}
       \label{fig:d2q9}
\end{figure}

\begin{equation}
\label{equ:d2q9}
    \vec{e}_{i} = \left\{\begin{matrix}
(0,0) \qquad\qquad\qquad\qquad\qquad\qquad &i=0 \\ 
(1,0), (0,1), (-1,0), (0,-1)\quad\qquad &i=1,2,3,4 \\ 
(1,1), (-1,1), (-1,-1), (1,1)\qquad & i=5,6,7,8
\end{matrix}\right.
\end{equation}

For each lattice, the distribution function can be thought of as a measure of the probability that the fluid at given position $\mathbf{x}$ has velocity $\mathbf{e}_i$ at time $t$.\\

As introduced above, in lattice Boltzmann method, the behaviour has been described as the streaming and collision step which are given by Equation~\ref{equ:lbmequ}. In this equation, $fi$ is the distribution function in the direction i; $\tau$ is the relaxation factor (reverse time); $f_i^{eq}$ is the equilibrium distribution function in the direction i; $t$ and $\triangle t$ is the time and the increment of time.\\

\begin{equation}
\label{equ:lbmequ}
f_i(\vec{x}+c\vec{e_i}\triangle t, t+\triangle t) - f_i(\vec x,t)) = \frac{f_i(\vec x, t) - f_i^{eq}(\vec x , t)}{\tau}
\end{equation}



% KS: EXPLAIN ALL SYMBOLS WHEN THEY ARE FIRST INTRODUCED!

In Equation~\ref{equ:lbmequ}, in the left of the equal sign stand for the streaming step and the collision step is in the right. In the actual implementation of our lattice Boltzmann method, we will calculate them separately. The Fig.~\ref{fig:stream} shows how the streaming step exchange its discrete probability distribution.\\


\begin{figure}[!tb]
   \centering
       \includegraphics[width=1\textwidth]{figures/stream.jpg}
       \caption{The streaming step of a lattice in D2Q9 model}
       \label{fig:stream}
\end{figure}

In the collision step, Bhatnagar-Gross-Krook (BGK)\ref{sec:BKG} approximation is applied to model the process of relaxation to equilibrium. Qian et al represent the equilibrium distribution function in their the DdQm models as Equation~\ref{equ:collision}. Macroscopic density $\rho$, velocity $u$, weighting factor $\vec{\omega_i}$ and lattice speed $c$ are introduced to this process in Equation~\ref{equ:collision}:

\begin{equation}
\label{equ:collision}
    f_i^{eq}(\vec x, t) = \vec{\omega_i} \rho + \rho \vec{\omega_i} \left [  \frac{\vec{e_i}\cdot \vec u}{c^2} + \frac{(\vec{e_i}\cdot\vec u)^2 }{ 2 \cdot c^4 }-\frac{\vec u \cdot \vec u}{2 \cdot c^2} \right ]
\end{equation}

According to Qian et al.'s model of D2Q9, $\omega_i$ is given as:
\begin{equation}
    \vec{\omega_i} = \left\{\begin{matrix}
4/9 & i=0\\ 
1/9&\quad \quad i=1,2,3,4\\ 
1/36&\quad \quad i=5,6,7,8 
\end{matrix}\right.
\end{equation}

Macroscopic fluid density $\rho (\vec x, t)$ is defined in this model by Equation~\ref{equ:rho}:
\begin{equation}
\label{equ:rho}
    \rho (\vec x, t) = \sum_{i=0}^{8} f_i(\vec x,t)
\end{equation}

Macroscopic velocity $\vec u(\vec x, t)$ is defined in this model by Equation~\ref{equ:u}:
\begin{equation}
\label{equ:u}
    \vec u (\vec x, t) = \frac{1}{\rho} \sum_{i=0}^{8}cf_i\vec{e_i}
\end{equation}

After all, we can summarize the algorithm as follow:
\begin{itemize}
  \item [1] Initialize macroscopic fluid density$\rho$, macroscopic velocity $\vec u$, and discrete probability distribution function $f_i^{eq}$;
  \item [2] Compute the $\rho$, $\vec u$ with Equation~\ref{equ:rho} and Equation~\ref{equ:u};
  \item [3] Compute the equilibrium distribution $f^{eq}_i$ with Equation~\ref{equ:collision};
  \item [4] Collision Step: Update the distribution function $f_i$;
  \item [5] Streaming Step: move $f_i$ to the $f_i^*$ in all nine direction with regarding to $\vec{e_i}$
\end{itemize}


With different initial parameters, the lattice Boltzmann method can simulation different physical problem. As we have discussed in \ref{sec:Obj}, Minion and Brown's problem\cite{minion1997performance} was chosen as a test problem. We will now discuss how to set the initial parameters to assure we are discuss a same physical problem.\\

% KS: WHAT INITIAL PARAMETERS DO YOU HAVE? 

\subsection{Initial Parameter}
To assure that the implementation is simulating the Minion and Brown's problem\cite{minion1997performance}, we need a criteria. In this case, the Reynolds number can be the criteria. The Reynolds number is the ratio of inertial forces to viscous forces within a fluid which is subjected to relative internal movement due to different fluid velocities\cite{munson2013fluid}. It is used to measure the similitude between fluid dynamics. If we set our parameters to have the same Reynolds number as the case in Minion and Brown's problem, we can assure that we are simulating their problem. In the definition of the Reynolds number at Equation~\ref{equ:neynolds}, $U$ is the velocity, $L$ the length scale, and $\nu$ is the kinematic viscosity.\\
\begin{equation}
\label{equ:neynolds}
    Re = \frac{U \cdot L}{\nu}
\end{equation}

In the Minion and Brown's problem, the initial velocity $u$ is defined as Equation~\ref{equ:init_u}, where $k$ is the shear layer thickness; $x$ and $y$ are the position in the system ; $\delta$ is a velocity. \\

\begin{equation}
\label{equ:init_u}
    \begin{matrix}
    
u = \mathrm{tanh}(k (y-0.25)) & y \leqslant 0.5 \\ 
u = \mathrm{tanh}(k (0.75-y)) & y > 0.5  \\
v = \delta sin(2\pi x )
\end{matrix}
\end{equation}

\begin{figure}[!tb]
   \centering
       \includegraphics[width=0.9\textwidth]{figures/double_c_256.png}
       \caption{Vorticity contour from the Minion and Brown's problem\cite{minion1997performance} setting  for resolution 128 $\times$ 128, number of contours being 20. Layer width parameter $\rho = 30$, viscosity $v=1/10,000$. }
       \label{fig:minion}
\end{figure}
In the parameters setting in Fig. 4 of Minion and Brown's work (seen Fig.~\ref{fig:minion}), there is a square system with a scale as $L_x = L_y = 1 unit = L$. The shear layer thickness $k$ they used is 30unit. Kinematic viscosity $\nu$ is $10^{-4}$. If we take $U = U_0 delta = 0.05$, $\nu$ = $10^{-4}$, and $L = 1$, then the corresponding Reynolds number is According to Equation~\ref{equ:neynolds}, we get the corresponding Reynolds number being 500, at Equation~\ref{Re_minion}:
\begin{equation}
\label{Re_minion}
    Re = UL / \nu = 0.05 x 1 / 10^{-4} = 500
\end{equation}

After calculate the Reynolds number of the Minion and Brown's work. We can adjust the initial parameters making the Reynolds the same in our implementation. Here, we set the simulation as $L_x = L_y = 128 = L$. The grid space is always $h = 1 unit$. The We now can apply the same initial velocities as Equation~\ref{equ:init_u_lbm}:

\begin{equation}
\label{equ:init_u_lbm}
    \begin{matrix}
u = U_0\mathrm{tanh}(k (y-0.25L)) & y \leqslant 0.5 \\ 
u = U_0\mathrm{tanh}(k (0.75L-y)) & y > 0.5  \\
v = \delta sin(2\pi x )
\end{matrix}
\end{equation}

Here, we should choose an equivalent value of shear layer thickness $k$; and, according to lattice Boltzmann method, the velocity unit $U_0$ should be less than the speed of sound $c_s$ and we choose $U_0 = 0.01$. To keep the Reynolds number the same, we can get the kinematic viscosity $\nu$ being 0.000128 by Equation~\ref{equ:nu_lbm}.\\

\begin{equation}
    \label{equ:nu_lbm}
    \nu = \frac{UL}{Re} = \frac{0.0005 \times 128}{500}  
\end{equation}

Finally, according to Minion and Brown's work, when T=0.8, there is a turbulence in the simulation; see Fig.~\ref{fig:minion}. We can calculate the time step (dimensionless time) by Equation\ref{timestep}. Substitute data from Minion and Brown's setting: time T = 0.8;  unit space $h=1/L=1/128$; unit velocity $U_0 \delta=U=0.05$  into the Equation\ref{timestep}, we know that in this system it need 2000 dimensionless time for the turbulence.\\

\begin{equation}
\label{timestep}
    T_{step} = T / \frac{h}{U_0 \delta} = \frac{0.8 \times 0.05}{ \frac{1}{128}}= 5.12
\end{equation}

In the lattice Boltzmann, we have unit space $h=1$, unit velocity $U_0 \delta = U = 0.0005$. Thus, substitute data into Equation~\ref{timestep}, we get $T_{step} = 2000 (timesteps)$, which means for every dimensionless time unit in Minion and Brown's system, we need 2000 steps to simulate to the same progress. In other words, to witness a turbulence in the lattice Boltzmann system, we need at least $5.12 \times 2000 = 10240$ steps. In the implementation and performance evaluation, we used 12000 steps as a standard; see Fig.~\ref{fig:minion}.\\

A more detailed initialization in serial C used for this project is at Appendix \ref{app:a}.\\


\vspace*{+3.2cm}
Having covered all the background material needed, we can now review the implementations that constitute one of the contributions of this project.


\newpage


\newpage
\section{Design and Implementation} \ref{sec:dai}
This section will be discussing the main contribution of this project: a implementation of lattice Boltzmann method on SpiNNaker.  \\

All the source code is available at: \url{https://github.com/YuaNFrank/SpiNN_LB}.
\subsection{Implementation Overview}

As we discussed at \ref{sec:sdw}, in general SpiNNaker development workflow, the first thing we did is to define the computation graph in Python. We will discuss how we define the lattice vertex (\ref{sec:tlc}), how we define our edge (\ref{sec:tle}), and how we connect them together as a graph (\ref{sec:tcg}). All these steps are implemented in Python API.
1       
Then on the computation side, as discussed at \ref{sec:lbmb}, there are five step for a lattice Boltzmann method in general. While the first four steps do not involve communication, and can be calculate locally for each lattice vertex (discussed at \ref{sec:cp}), the last step, the streaming step, is where the communication happens, which we will focus on at \ref{sec:ssc}. These computation-wise code is written in C.

\subsection{The Lattice Cell} \label{sec:tlc}
\subsubsection{Design Consideration} \label{sec:tlcdc}
In the lattice cell class, we will define the data regions needed for simulation. In implementation, it need to acquire those data, reserve memory for the data and pass the data in SpiNNaker SDRAM in Python, then define how to read the data from the SDRAM during simulation in C. The data regions that a lattice might need for simulation are: \\

\begin{itemize}
    \item \textbf{System Information:} for every simulation, they at least need some memory reserved for the runtime system such as the how long is a machine time step. And the developers need to generate the system data region and allocate memory for them manually.
    
    \item \textbf{Transmission Information:} for non-embarrassingly parallel problems, the simulation are involved in communication. The keys are generated by the runtime, and the developers can ask the runtime for a fix number of keys for every lattice. The developers need to allocate memory for the transmission keys and pass the keys to the cores, correspondingly.

    \item \textbf{Position Information:} in lattice Boltzmann, a lattice might need to know what is its position \textbf{(x,y)} among the whole simulation lattices. The position information would be generated when connecting the lattices as a graph, and then pass them into the application.
    
    \item \textbf{Initialized Velocity:} in lattice Boltzmann method, the velocity need to be generated according to the specific problem. We can either initialize it in Python then pass then in or initialize it in the simulation according to the position. A more detailed discussion is at \ref{sec:ip}.
    
    \item \textbf{Routing Information of Neighbours:} in lattice Boltzmann method, the lattice need to exchange the momentum and energy by moving the distribution function with its eight neighbours. This involve a few communication and the lattice need to know which are its eight neighbours and their routing information i.e. routing keys to communicate with them. Thus allocating memory and pass them into the cores is necessary. 
    
    \item \textbf{Index of this Lattice:} we might need to know the index of the lattice in the whole simulation fabric. This might not be necessary for a simple prototype. Since the random number generation is relatively slow, we can use it as a random number in practice. A further discuss is at \ref{sec:ssc}.
    
    \item \textbf{Result Recording:} the SDRAM in SpiNNaker is relatively limited (\ref{sec:sa}). We can use the recorded data to store larger simulation result more reliably. We can specify the an area of memory used for recording the results and get them back when the simulation ends.
\end{itemize}

\subsubsection{Final Implementation}
In the final implementation, we defined the discussed data regions in the \textit{LatticeBasicCell} class following the pattern: get the data; allocate memory; write the data in. For different data, there are different way to get the data:

\begin{itemize}
    \item \textbf{System Information:} the \textit{SpiNNaker GraphFrontEnd} provide API to generate the system data region.
    \item \textbf{Transmission Information:} the SpiNNaker runtime would allocate keys for the cores.
    \item \textbf{Position Information:} the position information is decided during runtime. it will be passed as class variables.
    \item \textbf{Initialized Velocity:} the velocities are decided during runtime with a initialization function. They will be passed as class variables. 
    \item \textbf{Routing Information of Neighbours:} a lattice can know the which are its neighbours by asking the connected edges. We will illustrate how we implement it at \ref{sec:tle}. After knowing its neighbours, we can get their routing keys via the get\_routing\_info\_from\_pre\_vertex() function from the \textbf{PACMAN} library.
    \item \textbf{Index of Lattice:} we can ask the graph system for the index of a lattice.
    \item \textbf{Result Recording:} we do not need to get data from result recording since it is for the result.
\end{itemize}

After all the data are acquired, we get then reserve corresponding memory and write the data into the SDRAM of each SpiNNaker core; see the upper part of Fig.~\ref{fig:write_data}. Correspondingly, in the C file, we also need to read the data from the SDRAM before the simulation; see the bottom part of Fig.\ref{fig:write_data}. The Fig.~\ref{fig:write_data} shows how we reserve memory for the different data regions and write the data into the SDRAM followed by how to read the written data from the SDRAM during runtime in C.
\begin{figure}[tb]
   \centering
       \includegraphics[width=1\textwidth]{figures/write_data.png}
       \caption{A lattice cell A connected by another lattice B via a lattice edge with the compass being "W". The lattice A then can get the routing key from its pre\_vertex, B.}
       \label{fig:write_data}
\end{figure}



\subsection{The Lattice Edge} \label{sec:tle}
\subsubsection{Design Consideration}
The major job of the lattice edge except connecting two lattice cells is providing a way that the lattice cell can get the vertex on the other end of this edge in a given direction. So that each lattice can get the routing information (routing keys) of its eight neighbours with knowing the direction. 
\subsubsection{Final Implementation}
In the implementation, the \textit{Lattice Edge class} inherit from the \textit{MachineEdge}, which provided a way to record the pre-vertex and post-vertex, from the \textbf{PACMAN} library. Then we introduced another attribute, \textbf{compass}, to record the relative position of the pre-vertex; see Fig.~\ref{fig:edge}.
\begin{figure}[tb]
   \centering
       \includegraphics[width=0.7\textwidth]{figures/edge.jpg}
       \caption{A lattice cell A connected by another lattice B via a lattice edge with the compass being "W". The lattice A then can get the routing key from its pre\_vertex, B.}
       \label{fig:edge}
\end{figure}

\subsection{The Computational Graph} \label{sec:tcg} 
\subsubsection{Design Consideration}
In the selected test problem from Minion and Brown\cite{minion1997performance}, the boundary condition we applied for this project is a periodic condition. In the C implementation, we implemented the periodic condition is halo-swapping; see Fig.~\ref{fig:haloswap}. The periodic condition is explicitly applied to the post-collision distribution function $f_i^{*}$ in the streaming step, which introduced extra memory consumption and complexity into the development.

Fortunately, in SpiNNaker, the developers do not implement the halo-swapping to apply to the periodic condition explicitly. Instead, SpiNNaker developers can simply connect the lattices on one fringe to the lattices on the opposite fringe via the implemented \textit{Lattice Edge}; see Fig.~\ref{fig:spinnaker_halo}.


\begin{figure}[!tb]

\begin{subfigure}[b]{1\textwidth}
       \centering
       \includegraphics[width=0.8\textwidth]{figures/haloswap.png}
       \caption{A periodic condition in M$\times$N lattice Boltzmann implemented by halo-swapping. We apply this method in our serial C implementation.}
       \label{fig:haloswap}
   \end{subfigure}
   \begin{subfigure}[b]{1\textwidth}
       \includegraphics[width=\textwidth]{figures/2dfabric.png}
       \caption{A periodic condition in 3$\times$3 lattice Boltzmann implemented by connecting the lattice periodically via lattice edge. We apply this method in the SpiNNaker implementation.}
       \label{fig:spinnaker_halo}
   \end{subfigure}
\end{figure}

\subsubsection{Final Implementation}
As we designed, to implement a periodic condition, we connect the lattice cells periodically via lattice edge; see Algo.~\ref{algo:periodic}.

\begin{algorithm}
 \caption{The Algorithm to connect the lattice with a periodic condition}
 \label{algo:periodic}
 \KwData{Scale in x dimension: X; Scale in y dimension: Y}
 \For{i in 0...X}{
    \For {j in 0...Y} {
       
    }
 }

\end{algorithm}




\subsection{Initialize Parameters} \label{sec:ip}
\subsubsection{Design Consideration}
To initialize the parameters for the lattice Boltzmann, there are two different ways in general. The first one is initialization on host (in Python) then passing the initialized parameters into the simulation, and the other one is initialization on SpiNNaker (in C).\\

When initializing parameters in Python on host, we need to compute the parameters in serial nested loops before the SpiNNaker actually run the simulation. On contrast. If we initializing the parameters in SpiNNaker, the parameters would be computed distributively on each individual core. Beside, vanilla Python is commonly regard as being poor in scientific computing, especially in nested loop.\\

It is obvious that initialization in C on the simulation runtime would be faster in speed comparing with initialization in Python in serial. Thus, our first implementation was based on C and compute the initial parameters distributively on each core. \\

However, there are two points in SpiNNaker that we do not take into consideration. The first one is that, as we introduced before, SpiNNaker is using ARM968 cores which have no floating point hardware and we have to use fixed-point arithmetic or a software-based floating-point arithmetic. It introduced some challenges in using mathematical library in C including \textbf{math.h}. The other one is that, as we also introduced before, the SpiNNaker cores have relatively limited ITCM (instruction tight-coupled memory). If we are plan to use mathematical library such as \textbf{math.h}, the library would run out of the limited ITCM. Although it is possible to implement mathematics functions such as \textbf{tanh} by applying the Taylor series, the accuracy and efficiency of the hand-write functions do not satisfy the requirements of this simulation.\\

After carefully reconsider the whole process, we finally decided to implement the initialization on the host in Python and then pass the initialized parameters as data regions to the SpiNNaker device.
\subsubsection{Final Implementation}


\subsection{Computation Implementation} \label{sec:cp}
\subsubsection{Design Consideration}
\subsubsection{Final Implementation}

\subsection{Streaming Step -- Communication Implementation} \label{sec:ssc}
\subsubsection{Design Consideration}
\subsubsection{Implementation}

\subsection{Communication Optimization}
\subsubsection{Design Consideration}
\subsubsection{Final Implementation}
\newpage
\section{Result and Evaluation} \label{sec:eval}

\newpage
\section{Conclusion} \label{sec:conclusion}
\subsection{Summary} \label{sec:summary}




% In this report, we have illustrate how SpiNNaker hardware can be applied to Computational fluid dynamics problem. More specifically, a basic implementation as well as some communication optimization of lattice Boltzmann method has been discussed. After confirm the correctness and quantify the accuracy of our implementation, we observe a good weak-scaling with the SpiNNaker hardware. Though there is still some limitation, this project build a solid foundation for future's exploration of porting CFD models into SpiNNaker platform.
% What was easy? What was difficult (compared to e.g., the CPU
% version)? Is SpiNNaker really a good platform for running
% such a simulation? Is it easy to develop and debug? Is the
% communication easy to arrange? Is it easy to measure and
% understand performance? Lattice Boltzmann is relatively
% simple (it was selected for this project to be so). What
% might happen if a more complex model was required?




\subsection{Future Work} \label{sec:future_work}
SpiNNaker is a good platform for running CFD simulation. In this project, we have not yet engaged its full capacity. There are basically two ways to go further.

The first one is on the CFD side, in this project, we only implement the most basic case in LBM. There are different periodic condition, optimization method yet to implement. Beside, there are still many CFD models that can benefit from SpiNNaker's massive parallelism. Therefore, in the future, more CFD models can be ported to SpiNNaker for performance. 

The second one is on the SpiNNaker side, in this project, we only allocate one lattice per core. Therefore, the simulation scale is very limited. In theory, a SpiNNaker core is able to run over 256 lattices and in the near future this number would be over 1,000. In this case, the SpiNNaker will benefit to with real-world engineering problems.





\newpage
\appendix
\section{Appendix - Initialization function in Serial C} \label{app:a}
\begin{minted}{C}
double U_0 = 0.01;
double K = 30.0;
double delta = 0.05;
/*
 * Init the velocity u, v = (u_x, u_y)
 * 
 * ydim = xdim = N = 128
 * K = 30 / N
 * delta = 0.05
 * 
 * u = tanh( K (y - 0.25 * ydim) ) for y <= 0.5 * ydim 
 * u = U_0 tanh( K (0.75 * ydim - y) ) for y >  0.5 * ydim
 * 
 * v = delta sin( 2pi *x / xdim )
*/
void initRho_N_U(double (*u_x)[ydim + 2], double (*u_y)[ydim + 2], double (*rho)[ydim + 2])
{
    for (int i = 1; i < xdim + 1; i++)
    {
        for (int j = 1; j < ydim + 1; j++)
        {
            double x_temp, y_temp;
            x_temp = 1.0 * (double)(i - 1) / (double)xdim;
            y_temp = 1.0 * (double)(j - 1) / (double)ydim; // scale the i and j to make 0 < i,j < 1

            if (y_temp <= 0.5)
            {
                u_x[i][j] = U_0 * tanh(K * (y_temp - 0.25));
            }
            else
            {
                u_x[i][j] = U_0 * tanh(K * (0.75 - y_temp));
            }

            u_y[i][j] = U_0 * delta * sin(2.0 * M_PI * (x_temp + 0.25));
            rho[i][j] = 1.0;
        }
    }
}
\end{minted}

\newpage
%\bibliographystyle{alpha}
%\bibliographystyle{ieeetr}
\bibliographystyle{IEEEtran}
\bibliography{bibs/references}
\newpage
\listoffigures

\newpage
\listoftables

\end{document}